\documentclass{article}
\usepackage[utf8]{inputenc}

\title{MATH3160 — Discrete Case Portfolio}
\author{Mike Medved}
\date{December 7th, 2022}

\usepackage{color}
\usepackage{amsthm}
\usepackage{amssymb} 
\usepackage{amsmath}
\usepackage{mathtools}
\usepackage{listings}
\usepackage[margin=1in]{geometry} 
\usepackage[dvipsnames,table]{xcolor}

\newtheorem*{thm}{Theorem}

\begin{document}

\maketitle

\section{Deliverables}

\subsection{Joint Distribution Function}

Let $X, Y$ be two discrete random variables for the sake of the below definitions and exercises.

\subsubsection{Joint Distribution Function}

The joint distribution function of $X$ and $Y$ is defined on $F \colon \mathbb{R}^2 \rightarrow [0,1]$, with the following formula:

\begin{equation*}
    F(x,y) = P(X \leq x, Y \leq y)
\end{equation*}

\subsubsection{Joint Probability Mass Function}

The joint pmf of $X$ and $Y$ is defined on $f \colon \textit{Im X} \times \textit{Im Y} \rightarrow [0,1]$, and takes the following formula:

\begin{equation*}
    f(x,y) = P(X = x, Y = y) \quad \forall (x,y) \in \textit{Im X} \times \textit{Im Y}
\end{equation*}

\subsubsection{Independence of $X$ and $Y$}

$X$ and $Y$ are said to be independent if their pmf functions hold the following: $f(x,y)=f_X(x) \cdot f_Y(y)$. However, in order to be independent, this property must hold $\forall (x,y) \in \textit{Im X} \times \textit{Im Y}$.

$\hfill \break$
Equivalently, $X$ and $Y$ are also said to be independent if their distribution functions hold the following property, $F(x,y) = F_X(x) \cdot F_Y(y)$.

\subsubsection{Examples}

Let us throw two fair dice, the random variable $X$ will represent the number of sixes rolled between the two dice, and $Y$ will represent the number that appeared on die 1.

$\hfill \break$
The sample space $S$ for an experiment containing two dice rolls is the following set $S$, with the pmf table:

$$
    S = \{(1, 1), (1, 2), (1, 3), (1, 4), (1, 5), (1, 6), ..., (6, 1), (6, 2), (6, 3), (6, 4), (6, 5), (6, 6)\}
$$

\begin{table}[h]
    \centering
    \begin{tabular}{l|l|l|l|l|l|l|l}
    \cline{2-8}
                                                                & \textbf{1}                  & \textbf{2}                  & \textbf{3}                  & \textbf{4}                  & \textbf{5}                  & \textbf{6}                  & \multicolumn{1}{l|}{\cellcolor[HTML]{67FD9A}\textbf{pmf X}} \\ \hline
    \multicolumn{1}{|l|}{\textbf{0}}                             & $\frac{5}{32}$                        & $\frac{5}{32}$                        & $\frac{5}{32}$                        & $\frac{5}{32}$                        & $\frac{5}{32}$                        & 0                           & \multicolumn{1}{l|}{\cellcolor[HTML]{67FD9A}$\frac{25}{32}$}          \\ \hline
    \multicolumn{1}{|l|}{\textbf{1}}                             & $\frac{1}{32}$                        & $\frac{1}{32}$                        & $\frac{1}{32}$                        & $\frac{1}{32}$                        & $\frac{1}{32}$                        & $\frac{5}{32}$                        & \multicolumn{1}{l|}{\cellcolor[HTML]{67FD9A}$\frac{10}{32}$}          \\ \hline
    \multicolumn{1}{|l|}{\textbf{2}}                             & 0                           & 0                           & 0                           & 0                           & 0                           & $\frac{1}{32}$                        & \multicolumn{1}{l|}{\cellcolor[HTML]{67FD9A}$\frac{1}{32}$}           \\ \hline
    \multicolumn{1}{|l|}{\cellcolor[HTML]{67FD9A}\textbf{pmf Y}} & \cellcolor[HTML]{67FD9A}$\frac{1}{6}$ & \cellcolor[HTML]{67FD9A}$\frac{1}{6}$ & \cellcolor[HTML]{67FD9A}$\frac{1}{6}$ & \cellcolor[HTML]{67FD9A}$\frac{1}{6}$ & \cellcolor[HTML]{67FD9A}$\frac{1}{6}$ & \cellcolor[HTML]{67FD9A}$\frac{1}{6}$ &                                                             \\ \cline{1-7}
    \end{tabular}
\end{table}

$\hfill \break$
According to the definition of joint independence, we can see that $X$ and $Y$ are independent, since the following property holds: $f(0,1) = \frac{25}{32} \cdot \frac{1}{6} = \frac{5}{32}$, and $f(1,1) = \frac{10}{32} \cdot \frac{1}{6} = \frac{1}{32}$. Therefore $X$ and $Y$ must be independent.

\end{document}