\documentclass{article}
\usepackage[utf8]{inputenc}

\title{MATH3160 — Continuous Case Portfolio}
\author{Mike Medved}
\date{December 10th, 2022}

\usepackage{color}
\usepackage{amsthm}
\usepackage{amssymb} 
\usepackage{amsmath}
\usepackage{mathtools}
\usepackage{listings}
\usepackage[margin=1in]{geometry} 
\usepackage[dvipsnames,table]{xcolor}

\newtheorem*{thm}{Theorem}

\begin{document}

\maketitle

\section{Deliverables}

\subsection{Joint Continuous Random Variables}

Let $X, Y$ be two discrete random variables for the sake of the below definitions and exercises.

\subsubsection{Joint Distribution Function}

The joint distribution function of $X$ and $Y$ is defined on $F \colon \mathbb{R}^2 \rightarrow [0,1]$, with the following formula:

\begin{equation*}
    F(x,y) = P(X \leq x, Y \leq y) = \int_{-\infty}^x \int_{-\infty}^y f(u, v) \, \mathrm{d}v \, \mathrm{d}u
\end{equation*}

\subsubsection{Joint Density Function}

The joint density of $X$ and $Y$ is defined on $f \colon \mathbb{R}^2 \rightarrow [0,\infty]$, and takes the following formula:

\begin{equation*}
    f(x,y) = P((x, y) \in \mathbb{R}^2) = \iint_D f(x,y) \, \mathrm{d}x \, \mathrm{d}y \text{ } \forall D \in \mathbb{R}^2
\end{equation*}

\subsubsection{Independence of $X$ and $Y$}

$X$ and $Y$ are said to be independent if their marginal densities hold equal their joint density, as such: $f(x,y)=f_X(x) \cdot f_Y(y)$. However, in order to be independent, this property must hold $\forall (x,y) \in \mathbb{R}^2$.

$\hfill \break$
Equivalently, $X$ and $Y$ are also said to be independent if their marginal distributions hold the following property, $F(x,y) = F_X(x) \cdot F_Y(y) \text{ } \forall (x,y) \in \mathbb{R}^2$.

\subsubsection{Importance of the Density Function}

The density function for a pair of continuous random variables presents an remarkable property, that it's integral over the entire space is equal to 1.

$\hfill \break$
This property holds because the integral $\iint_D f(x,y) \, \mathrm{d}x \, \mathrm{d}y$ over the domain $D$ represents the volume of the sample space, and due to the law of total probability, the volume of the sample space must be equal to 1.

\subsubsection{Marginal Densities}

The marginal density of $X$ is defined as $f_X(x) = \int_{x}^{1} f(x,y) \, \mathrm{d}y$, and the marginal density of $Y$ is defined as $f_Y(y) = \int_{y}^{1} f(x,y) \, \mathrm{d}x$.

\end{document}